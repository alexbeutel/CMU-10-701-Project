\documentclass{article}

\usepackage{amsmath} % loads AMS-Math package
\usepackage{euscript}
\usepackage{amssymb}
\usepackage{epsfig} % allows PostScript files
\usepackage{graphicx}
\usepackage{listings} % allows lstlisting environment
\usepackage{moreverb} % allows listinginput environment
\usepackage{vmargin} % allows better margins
\usepackage{color}
%\usepackage{hyperref}
\setpapersize{USletter} % sets the paper size
\allowdisplaybreaks[1]
\setmarginsrb{1in}{0.7in}{1in}{1in}{12pt}{11mm}{0pt}{11mm} %sets margins

\newcommand{\alex}[1]{\textcolor{red}{#1}}

%\setlength{\parskip}{3mm} %set paragraph spacing
%\setlength{\parindent}{0mm} %set indent
%\renewcommand{\labelenumi}{\Roman{enumi}.}
%\renewcommand{\labelenumii}{\Alph{enumii}.}

\begin{document}
\noindent\begin{tabular*}{6.5in}{@{}l@{\extracolsep{\fill}}r@{}}
		{\sc {\Large Project Proposal}} & Nicolas Feltman and Alex Beutel \ $\cdot$\ \ March 26, 2012 
	\end{tabular*}\\
	\rule[3mm]{6.5in}{0.1mm}

	\noindent {\large {\bf Title: 3D Scene Construction with Noisy Data using Spectral Methods} }

	\subsection*{Data} % (fold)
	Data used will be video collected with an iPhone or Android phone.  As the
	goal is explained below, we would like to work with noisy data, so data
	acquired freely by consumer devices such as these should suffice.
	
	% subsection Datasubsection name (end)

	\subsection*{Project Idea} % (fold)
	\label{sub:Project Idea}

	The goal of this project is to be able to create accurate 3D models of an
	object from noisy video.  We hope to use a combination of techniques from
	the machine learning and computer vision communities, and particularly work
	closely with Professor Geoff Gordon.

	In order to construct 3D scenes from noisy video will require a variety of
	different methods.  From computer vision, we plan to use the SIFT
	(scale-invariant feature transform) algorithm to detect features in video,
	along with other methods by Carlo Tomasi et al. to track features.  We then
	hope to combine these methods with a Hidden Markov Model and spectral
	methods, based on recent work by Professor Gordon, to model the movement of
	the points over time, deduce the motion of the camera, and determine the 3D
	shapes of the scene.

	
	% subsection Project Idea (end)

	\subsection*{Software} % (fold)
	\label{sub:Software}

	In meeting with Professor Gordon, he said he had code from his research
	that performs some of the spectral methods used for his research and that
	he was planning to open source it this week.  We plan to build on his
	code-base, incorporating SIFT techniques and making more robust software
	for 3D scene construction.  The direction of work here will depend on how
	robust his code is to start.  If it is already very robust for 3D scene
	construction, we hope to expand it further.
	
	% subsection Software (end)

	%\subsection*{Papers} % (fold)
	%\label{sub:Papers}

	\begin{thebibliography}{9}

		\bibitem{boots}
			Boots, Byron and Geoffrey J. Gordon.
			\emph{An Online Spectral Learning Algorithm for Partially Observable Nonlinear Dynamical Systems}.
			AAAI 2011.

		\bibitem{lowe2004}
			Lowe, David G.
			\emph{Distinctive Image Features from Scale-Invariant Keypoints}.
			International Journal of Computer Vision 60, 2. November 2004.

		\bibitem{lowe1999}
			Lowe, David G.
			\emph{Object Recognition from Local Scale-Invariant Features}.
			1999.

		\bibitem{tomasi}
			Tomasi, Carlo and Takeo Kanade.
			\emph{Detection and Tracking of Point Features}.
			Shape and Motion from Image Streams: A Factorization Model.
			Technical Report CMU-CS-91-132,
			April 1991.

	\end{thebibliography}
	
	% subsection Papers (end)

	\subsection*{Collaboration} % (fold)
	\label{sub:Collaboration}

	The project will be a collaboration between Nicolas Feltman and Alex
	Beutel.  Since much of the source that we'll be working with has not yet been released,
	we are not entirely sure how yet to divide the coding.  Regardless, we would both like to 
	play an active role in all parts of the work, and we will split it up further as we get a 
	better sense of the work required.  Nicolas is a student	in graphics and is generally 
	familiar with vision problems and spectral methods, whereas Alex is familiar with vision and HMMs.

	We also plan to meet with Professor Geoff Gordon regularly to discuss the
	progress of the project, since it is closely related to his research.
	
	% subsection Collaboration (end)

	


\end{document}
