\section{Introduction} % (fold)
\label{sec:Introduction}

A common challenge in computer vision is accurately tracking features through a
video.  Many computer vision algorithms build upon these tracked features, so
accurately tracking the features and understanding the traces through time that
the features produce is important.  There has been much prior work on feature
selection (ie SIFT \cite{sift}), which focuses on finding features in an image
and picking out the ``best'' ones.  Important later work focused on matching
features between images and after that tracking features through a sequence of
frames in a video (ie KLT \cite{klt}).  These tools produce a trace of the
features through space and time.
Given these traces for each feature, researchers attempt to use them for
numerous computer vision applications; we focus on two: robust structure from
motion and robust movement isolation.    

Structure from motion is a common application in which we attempt to recover
the structure of the objects in a scene based on the motion of a camera shown
in the feature traces.  (This problem will be discussed in more detail
later.)  However, bad feature traces \alex{worsen} the results of the structure
from motion algorithm.
In particular, when running SIFT and KLT, we found that the algorithms had some
difficulty with shaky video, such as that produced from waving a mobile phone
around.  Also, we found that if there was a moving object in a scene we filmed,
the traces of the moving object also weakened the structure from motion
algorithm. 
Therefore, we sought to create a model by which to separate these abnormal
feature traces from the normal, valid ones, with the ultimate goal of removing
abnormal traces and improving the structure from motion.  Our key insight,
which we will explain in more detail later, is that normal traces of static
objects in a scene generally look similar to one another since their trace is
almost entirely based on the motion of the camera, where as the bad traces and
traces of moving objects look very different.  Therefore, we created a model to
compare trace shapes and detect outliers.

As we produced this model, we found it was excellent at detecting moving
objects in a scene, regardless of camera motion.  We decided to apply our
outlier detection model to robustly isolating motion in shaky video.  Given an
``outlier score'' from our previous model we create a secondary model to
estimate if a given pixel in a given frame of video is of a moving object or a
static objects, based on the traces around that pixel and their scores.  Under
this model, we can then construct a video containing only the moving elements
in the video and removing background pixels.

In Section \ref{sec:background} we give a brief background on the computer
vision research that we build upon, such as SIFT, KLT, and structure from
motion.  In Section \ref{sec:formulation} we more precisely formulate the
problem and describe our model for outlier detection.  In Section \ref{sec:sfm}
we describe the application of our model to the problem of robust structure
from motion, and in Section \ref{sec:motion} we describe the application of our
model to the problem of robust motion isolation.  For each application we run
our algorithm on numerous videos, some taken by us with mobile phones and
others downloaded from YouTube, and analyze the results.
