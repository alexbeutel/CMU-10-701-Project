\section{Robust Motion Isolation} 

As discussed previously, our Gaussian mixture model does an excellent job of
distinguishing traces of moving objects from traces of static objects caused
only by the motion of the camera (regardless of the camera motion).  We decided
to apply this aspect of our model to the task of isolating motion in a scene,
regardless of unstable video footage.  We first describe the model we use to
take traces and corresponding GMM probabilities and create a video with only the
moving elements.  We then describe the experiments and results of our
implementation.

\subsection{Model Application} % (fold)
\label{sub:Model Application}

\paragraph{Traces to Frames} % (fold)
\label{par:Traces to Frames}

Our problem can be formulated as follows: Given a set of traces and the GMM
probability for each, we look to find the probability that pixel $\pi_j =
(x_j,y_j)$ at time $t$ is of a moving object in the scene or not.  In order to
calculate this we first transform a few parameters.  First, we invert and bound
the range of GMM probabilities to create a \alex{movement} score.  Given ${\bf
P}$ is the vector of log-probabilities returned by the GMM for a given set of
traces, we create the following vector of movement scores:
\begin{align}
	{\bf s} = - \frac{ {\bf P} - {\rm max}({\bf P}) }{ {\rm min}({\bf P}) }
\end{align}
Under this transformation, the scores are bounded to the range 0 to 1 such that
traces with a high probability of being outliers, and thus movement, are close
to 1, and those that are not are closer to 0.

With this score, we can create a probabilistic model for any given pixel being
of a moving object or not.  In our model we assume independence among both
pixels and traces.  Therefore, given score $s_i$ for trace $T_i$, at position
$(x_{i,t},y_{i,t})$ at time $t$, we claim the following probabilities:
\begin{align}
	P(s_i,x_{i,t},y_{i,t}|{\rm pixel}_j = {\it moving})  &= \frac{1}{1 + e^{-\gamma(s_i - \beta)}} e^{- \frac{|(x_{i,t},y_{i,t}) - \pi_j|^2}{\alpha_1}}
	\\ P(s_i,x_{i,t},y_{i,t}|{\rm pixel}_j = {\it static})  &= \left( 1 - \frac{1}{1 + e^{-\gamma(s_i - \beta)}}\right) e^{- \frac{|(x_{i,t},y_{i,t}) - \pi_j|^2}{\alpha_2}}
\end{align}
The first term in each probability puts the movement score through the logistic
function in order to partition scores into being generally very close to zero
(most likely static) or very close to one (most likely moving).  For this
logistic function we have two parameters: $\beta$ and $\gamma$.  $\beta$ we
pick as some threshold score for which scores less than the threshold are
pushed to 0 and scores greater than the threshold are pushed to one.  Picking
this threshold is similar to the problem of picking the threshold for the
structure for motion problem.  While this could be done dynamically based on an
analysis of the score distribution, similar to that in Figure
\ref{fig:logpdfs}, we generally used a threshold at about the 75th percentile
mark to separate the scores.  (Remember that scores are inverted such that high
scores are more likely to be of moving objects.)  The parameters $\gamma$ is
used to push values toward zero or one and leave few values near 0.5; we set
$\gamma = 100$.

The second term is \alex{of the form of a Gaussian} using the distance between
the feature and the pixel to determine what influence the feature's score has
on the probability for the pixel.  The terms $\alpha_1$ and $\alpha_2$
determine the \alex{width} of the Gaussian.  With this function we make sure
that features closer to a pixel have a stronger influence on their probability
than pixels far away.  We also set, based on experimentation, $\alpha_1 =
2\alpha_2$ so that features that appear be moving having a wider range of
influence than those that do not.

Because of our independence assumptions, we can sum the logarithms of the
probabilities calculated on a per pixel basis.  At the of the calculation, if
$P({\rm pixel}_j = {\it moving}) > P({\rm pixel}_j = {\it static})$ then we
consider the pixel to be of a moving object, and if not we consider it to be
static.  Doing this for all pixels in each frame, we can construct a video
isolating only the moving elements.

\alex{is there a way to weave ``expectation maximization'' into this?}

% paragraph Traces to Frames (end)

\paragraph{Trace Sets} % (fold)
\label{par:Trace Sets}

One key challenge we found in performing feature tracking was that features
were often lost if tracked over many sets of frames.  In the case of motion
isolation, we do not care so much if any given feature is tracked continuously,
but rather just if each pixel is more likely to be of a moving or static
object.


\newcommand{\fset}{ \EuScript{F} }
\newcommand{\fstep}{ f_s }
\newcommand{\fwin}{ f_w }

Therefore, to deal with losing features, we break the video into shorter frame
sets, each frame set denoted by $\fset_k$.  In order to keep some of the
elements of tracking across the video, we have some overlap between frame sets,
which we denote as frame step $\fstep$.  Therefore, the first frame of
$\fset_k$ is $\fstep$ frames after the first frame of $\fset_{k-1}$. Frame
sets have a window size $\fwin$, such that they contain a total of
$\fwin\fstep$ frames.

Because we assume independence among traces, we can similarly 


In practice we set $\fwin = 3$ and $\fstep = 3$ or 5.


% paragraph Trace Sets (end)



% subsection Model Application (end)

\subsection{Experiments} % (fold)
\label{sub:Experiments}

% subsection Experiments (end)
