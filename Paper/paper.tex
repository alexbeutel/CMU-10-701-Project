\documentclass{article} % For LaTeX2e
\usepackage{nips12submit_e,times}
%\documentstyle[nips12submit_09,times,art10]{article} % For LaTeX 2.09


\title{Tracked Feature Outlier Detection}


\author{
Nicolas Feltman\\
Department of Computer Science\\
Carnegie Mellon University\\
Pittsburgh, PA 15213 \\
\texttt{nfeltman@cs.cmu.edu} \\
\And
Alex Beutel\\
Department of Computer Science\\
Carnegie Mellon University\\
Pittsburgh, PA 15213 \\
\texttt{abeutel@cs.cmu.edu} 
}

% The \author macro works with any number of authors. There are two commands
% used to separate the names and addresses of multiple authors: \And and \AND.
%
% Using \And between authors leaves it to \LaTeX{} to determine where to break
% the lines. Using \AND forces a linebreak at that point. So, if \LaTeX{}
% puts 3 of 4 authors names on the first line, and the last on the second
% line, try using \AND instead of \And before the third author name.

\newcommand{\fix}{\marginpar{FIX}}
\newcommand{\new}{\marginpar{NEW}}

\nipsfinalcopy % Uncomment for camera-ready version

\begin{document}


\maketitle

\begin{abstract}
In this project, we employ a naive Bayes Gaussian mixture model to detect outliers in a set of image-feature traces derived from SIFT and KLT analysis of unstable video data.  We identify two applications: 1) preprocessing data for standard structure-from motion algorithms, and 2) robust motion detection and isolation in presence of shakey video.  In the first application, we find improvement of up to 100\% in our decomposability metric for some scenes.  In the second application, we show a strong ability to disentangle scene motion from camera motion.
\end{abstract}


\section{Introduction} % (fold)
\label{sec:Introduction}
Just some sections listed below following general outlined given on the project
page, though this may not be the best flow for our project.
% section Introduction (end)

\section{Problem Formulation} % (fold)
\label{sec:Problem Formulation}

% section Problem Formulation (end)

\section{Methodology} % (fold)
\label{sec:Methodology}

% section Methodology (end)

\section{Experiments} % (fold)
\label{sec:Experiments}

% section Experiments (end)

\section{Conclusion} % (fold)
\label{sec:Conclusion}

% section Conclusion (end)



\subsubsection*{References}

References follow the acknowledgments. Use unnumbered third level heading for
the references. Any choice of citation style is acceptable as long as you are
consistent. It is permissible to reduce the font size to `small' (9-point) 
when listing the references. {\bf Remember that this year you can use
a ninth page as long as it contains \emph{only} cited references.}

\small{
[1] Alexander, J.A. \& Mozer, M.C. (1995) Template-based algorithms
for connectionist rule extraction. In G. Tesauro, D. S. Touretzky
and T.K. Leen (eds.), {\it Advances in Neural Information Processing
Systems 7}, pp. 609-616. Cambridge, MA: MIT Press.

[2] Bower, J.M. \& Beeman, D. (1995) {\it The Book of GENESIS: Exploring
Realistic Neural Models with the GEneral NEural SImulation System.}
New York: TELOS/Springer-Verlag.

[3] Hasselmo, M.E., Schnell, E. \& Barkai, E. (1995) Dynamics of learning
and recall at excitatory recurrent synapses and cholinergic modulation
in rat hippocampal region CA3. {\it Journal of Neuroscience}
{\bf 15}(7):5249-5262.
}

\end{document}
