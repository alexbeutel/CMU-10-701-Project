\section{Background}
\label{sec:background}

Our method is built on top of the SIFT \cite{sift} feature selector and KLT \cite{klt}\cite{kltImpl} feature matching algorithm.  These were chosen because they meet a minimum level of performance required to illustrate our method, and stable implementations of both are freely available.  The SIFT algorithm is generally desirable because it is invariant to changes in scale and lighting, as well as small changes in viewing angle.  The former properties may not be necessary given our particular application, and so investigation into other features selectors could reasonably improve our implementation.  The only notable property of the KLT matching algorithm is that it only returns features represented in every input frame.  In practice, this is only a problem when the video includes a ``bad frame'' (eg. one frame is extremely blurry) which scrambles all feature tracking.

One of our applications builds on structure from motion \cite{sfm}\cite{sfmImpl}, in which the task is to use the projection of a labeled set of three-dimensional points from several viewpoints (``frames'') to recover (1) the true location of the points in 3D space, and (2) the location and direction of the camera for each frame.  In order to support our work, we will cover some of the exposition of \cite{sfm}. One key simplifying assumption in structure from motion is orthography, that points are projected orthographically onto the plane of the camera.  Firstly, note that the centroid of a set of points commutes with orthographic projection of those points onto some plane.  Where $p_1, \ldots, p_P$ is a series of points in three dimensional space, and $A$ is any $2\times 3$ orthogonal projection matrix:
\begin{align}
{\it centroid}({\it project}_A (\{p_1, \ldots, p_P\})) = A[p_1 \ldots p_P]\left[\begin{array}{c}1/P \\ \vdots \\ 1/P\end{array}\right] ={\it project}_A({\it centroid} (\{p_1, \ldots, p_P\}))
\end{align}
This implies that we can correct for camera translation by normalizing every frame to the centroid of its points, thereby leaving only the effects of projection onto the camera plane.  Having done this, let $A_1$,$A_2$,\ldots,$A_F$ be the series of $2\times 3$ orthogonal projection matrices through time.  Thus a single 3D point $p$ will take the path $A_1p$, $A_2p$, \ldots, $A_Fp$ in normalized screen space.  Considering this across a whole set of normalized points $p_1$, $p_2$, \ldots, $p_P$, we get a grand matrix, $D$, of all data.  This can be factored into two seperate matrices, one corresponding to the direction of the camera and the other to the collection of points:
\begin{align}\label{decomp}
D=
\left[ \begin{array}{cccc} 
A_1p_1 & A_1p_2 & \cdots & A_1p_P \\
A_2p_1 & A_2p_2 & \cdots & A_2p_P \\
\vdots & \vdots & \ddots & \vdots \\
A_Fp_1 & A_Fp_2 & \cdots & A_Fp_P
 \end{array} \right ] = 
\left[ \begin{array}{c} A_1 \\ A_2 \\ \vdots \\ A_F \end{array} \right ] \left[ \begin{array}{cccc} p_1 & p_2 & \cdots & p_P \end{array} \right ]
\end{align}
Since the sizes of these factors are $2F \times 3$ and $3 \times P$, we know that the original $D$ matrix, absent any measurement noise, has rank 3.  Of course, there is more than one way to form the decomposition in Equation \ref{decomp}; extra constraints come from restrictions on the projection matrices, and remaining freedom is attributable to freedom of axis choice.  Details of the computation can be found in \cite{sfm}.  For our purposes, the important detail is that the $D$ is rank 3.
